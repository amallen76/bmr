\documentclass{ximera}

\title{Graphs and Functions}
\author{Amy Riordan}

\begin{document}
\begin{abstract}
Graphing and Functions video resources from Module 3.
\end{abstract}
\maketitle

\section*{Graphing and Functions}

Below each video, you’ll find a set of practice problems. Try those problems first. If you run into any difficulties, go back and rewatch the video, then attempt the problems again. This way, you’ll reinforce what you’ve learned and build a stronger understanding.

\section*{Graphing}

Graphing (ERAU|8:45)

\youtube{DDFWfU92yMc}

% filepath: /code/ModuleThree/graphingandfunctions.tex
% ...existing code...

\section*{Plotting Points on a Coordinate Plane}

\begin{problem}
Plot the point $(2, 3)$ on a coordinate plane. $\answer{(2, 3)}$
\begin{feedback}
Start at the origin. Move $2$ units to the right (x-axis), then $3$ units up (y-axis). Place the point at $(2, 3)$.
\end{feedback}
\end{problem}

\begin{problem}
Plot the point $(-4, 1)$ on a coordinate plane. $\answer{(-4, 1)}$
\begin{feedback}
Start at the origin. Move $4$ units to the left (x-axis), then $1$ unit up (y-axis). Place the point at $(-4, 1)$.
\end{feedback}
\end{problem}

\begin{problem}
Plot the point $(0, -5)$ on a coordinate plane. $\answer{(0, -5)}$
\begin{feedback}
Start at the origin. Do not move left or right (x = 0), then move $5$ units down (y-axis). Place the point at $(0, -5)$.
\end{feedback}
\end{problem}

% ...existing code...

% filepath: /code/ModuleThree/graphingandfunctions.tex
% ...existing code...

\section*{Graphing Lines on a Coordinate Plane}

\begin{problem}
Graph the line $y = 2x + 1$ on a coordinate plane. $\answer{y = 2x + 1}$
\begin{feedback}
Start by plotting the y-intercept at $(0, 1)$. Use the slope $2$ to go up $2$ units and right $1$ unit to plot another point at $(1, 3)$. Draw a straight line through these points.
\end{feedback}
\end{problem}

\begin{problem}
Graph the line $y = -x + 4$ on a coordinate plane. $\answer{y = -x + 4}$
\begin{feedback}
Plot the y-intercept at $(0, 4)$. The slope $-1$ means go down $1$ unit and right $1$ unit to $(1, 3)$. Draw a straight line through these points.
\end{feedback}
\end{problem}

\begin{problem}
Graph the line $y = 3$ on a coordinate plane. $\answer{y = 3}$
\begin{feedback}
This is a horizontal line. Plot points where $y = 3$ for any $x$ (e.g., $(0, 3)$, $(2, 3)$). Draw a straight horizontal line through these points.
\end{feedback}
\end{problem}

\begin{problem}
Graph the line $x = -2$ on a coordinate plane. $\answer{x = -2}$
\begin{feedback}
This is a vertical line. Plot points where $x = -2$ for any $y$ (e.g., $(-2, 0)$, $(-2, 4)$). Draw a straight vertical line through these points.
\end{feedback}
\end{problem}

\begin{problem}
Graph the line $y = \frac{1}{2}x - 1$ on a coordinate plane. $\answer{y = \frac{1}{2}x - 1}$
\begin{feedback}
Plot the y-intercept at $(0, -1)$. The slope $\frac{1}{2}$ means go up $1$ unit and right $2$ units to $(2, 0)$. Draw a straight line through these points.
\end{feedback}
\end{problem}

% ...existing code...

\section*{Functions}

Functions (ERAU|13:38)

\youtube{VMr7bHE_T8o}

% filepath: /code/ModuleThree/graphingandfunctions.tex
% ...existing code...

\section*{Using Function Notation}

\begin{problem}
If $f(x) = 2x + 5$, find $f(3)$. $\answer{11}$
\begin{feedback}
Substitute $x = 3$: $f(3) = 2(3) + 5 = 6 + 5 = 11$.
\end{feedback}
\end{problem}

\begin{problem}
If $g(x) = x^2 - 4x$, find $g(-2)$. $\answer{12}$
\begin{feedback}
Substitute $x = -2$: $g(-2) = (-2)^2 - 4(-2) = 4 + 8 = 12$.
\end{feedback}
\end{problem}

\begin{problem}
If $h(x) = 3x - 7$, find $h(0)$. $\answer{-7}$
\begin{feedback}
Substitute $x = 0$: $h(0) = 3(0) - 7 = 0 - 7 = -7$.
\end{feedback}
\end{problem}

\begin{problem}
If $k(x) = \dfrac{1}{x}$, find $k(4)$. $\answer{\frac{1}{4}}$
\begin{feedback}
Substitute $x = 4$: $k(4) = \frac{1}{4}$.
\end{feedback}
\end{problem}

\begin{problem}
If $m(x) = x^2 + 2x + 1$, find $m(-1)$. $\answer{0}$
\begin{feedback}
Substitute $x = -1$: $m(-1) = (-1)^2 + 2(-1) + 1 = 1 - 2 + 1 = 0$.
\end{feedback}
\end{problem}

% ...existing code...

% filepath: /code/ModuleThree/graphingandfunctions.tex
% ...existing code...

\section*{Graphing Functions}

\begin{problem}
Graph the function $f(x) = x^2$ on a coordinate plane. $\answer{f(x) = x^2}$
\begin{feedback}
Plot several points: $(0, 0)$, $(1, 1)$, $(-1, 1)$, $(2, 4)$, $(-2, 4)$. Connect the points with a smooth curve to form a parabola opening upward.
\end{feedback}
\end{problem}

\begin{problem}
Graph the function $f(x) = |x|$ on a coordinate plane. $\answer{f(x) = |x|}$
\begin{feedback}
Plot points: $(0, 0)$, $(1, 1)$, $(-1, 1)$, $(2, 2)$, $(-2, 2)$. Connect the points to form a "V" shape with the vertex at the origin.
\end{feedback}
\end{problem}

\begin{problem}
Graph the function $f(x) = x + 2$ on a coordinate plane. $\answer{f(x) = x + 2}$
\begin{feedback}
Plot the y-intercept at $(0, 2)$. Use the slope $1$ to plot another point at $(1, 3)$. Draw a straight line through these points.
\end{feedback}
\end{problem}

% ...existing code...

% filepath: /code/ModuleThree/graphingandfunctions.tex
% ...existing code...

\section*{Identifying Functions}

\begin{problem}
Is the equation $y = x^2$ a function? $\answer{Yes}$
\begin{feedback}
Yes, for every $x$ value there is exactly one $y$ value. Each input produces one output.
\end{feedback}
\end{problem}

\begin{problem}
Is the equation $x = y^2$ a function of $x$? $\answer{No}$
\begin{feedback}
No, some $x$ values (e.g., $x = 4$) correspond to two $y$ values ($y = 2$ and $y = -2$). This fails the vertical line test.
\end{feedback}
\end{problem}

\begin{problem}
Is the equation $y = |x|$ a function? $\answer{Yes}$
\begin{feedback}
Yes, each $x$ value gives exactly one $y$ value. The equation passes the vertical line test.
\end{feedback}
\end{problem}

% ...existing code...

% filepath: /code/ModuleThree/graphingandfunctions.tex
% ...existing code...

\section*{Evaluating Piecewise Functions}

\begin{problem}
Let $f(x) = \begin{cases} x+2 & \text{if } x < 0 \\ 3x & \text{if } x \geq 0 \end{cases}$. Find $f(-3)$.
$\answer{-1}$
\begin{feedback}
Since $-3 < 0$, use $f(x) = x+2$: $f(-3) = -3 + 2 = -1$.
\end{feedback}
\end{problem}

\begin{problem}
Let $g(x) = \begin{cases} 2x & \text{if } x < 5 \\ x-4 & \text{if } x \geq 5 \end{cases}$. Find $g(7)$.
$\answer{3}$
\begin{feedback}
Since $7 \geq 5$, use $g(x) = x-4$: $g(7) = 7 - 4 = 3$.
\end{feedback}
\end{problem}

\begin{problem}
Let $h(x) = \begin{cases} x^2 & \text{if } x \leq 2 \\ 5x & \text{if } x > 2 \end{cases}$. Find $h(2)$.
$\answer{4}$
\begin{feedback}
Since $2 \leq 2$, use $h(x) = x^2$: $h(2) = 2^2 = 4$.
\end{feedback}
\end{problem}

% ...existing code...

\end{document}
