\documentclass{ximera}

\title{Exponents and Scientific Notation}
\author{Amy Riordan}

\begin{document}
\begin{abstract}
Exponents and Scientific Notation video resources from Module 3.
\end{abstract}
\maketitle

\section*{Exponents and Scientific Notation}

Below each video, you’ll find a set of practice problems. Try those problems first. If you run into any difficulties, go back and rewatch the video, then attempt the problems again. This way, you’ll reinforce what you’ve learned and build a stronger understanding.

\section*{Exponents}

Exponents (ERAU|7:01)

\youtube{czhWyyBHRME}

% filepath: /code/ModuleThree/exponentsandscientificnotation.tex

\section*{Exponent Rules Practice}

\begin{problem}
Simplify $x^4 \cdot x^2$. $\answer{x^6}$
\begin{feedback}
When multiplying like bases, add the exponents: $x^{4+2} = x^6$.
\end{feedback}
\end{problem}

\begin{problem}
Simplify $\dfrac{y^7}{y^3}$. $\answer{y^4}$
\begin{feedback}
When dividing like bases, subtract the exponents: $y^{7-3} = y^4$.
\end{feedback}
\end{problem}

\begin{problem}
Simplify $(a^5)^2$. $\answer{a^{10}}$
\begin{feedback}
When raising a power to a power, multiply the exponents: $a^{5 \times 2} = a^{10}$.
\end{feedback}
\end{problem}

\begin{problem}
Simplify $m^0$. $\answer{1}$
\begin{feedback}
Any nonzero base raised to the zero power is $1$: $m^0 = 1$.
\end{feedback}
\end{problem}

\begin{problem}
Simplify $2^3 \cdot 2^4$. $\answer{128}$
\begin{feedback}
Add the exponents: $2^{3+4} = 2^7 = 128$.
\end{feedback}
\end{problem}

\begin{problem}
Simplify $\dfrac{5^6}{5^2}$. $\answer{625}$
\begin{feedback}
Subtract the exponents: $5^{6-2} = 5^4 = 625$.
\end{feedback}
\end{problem}

\begin{problem}
Simplify $(z^2)^5$. $\answer{z^{10}}$
\begin{feedback}
Multiply the exponents: $z^{2 \times 5} = z^{10}$.
\end{feedback}
\end{problem}

\begin{problem}
Simplify $p^{-3}$. $\answer{\dfrac{1}{p^3}}$
\begin{feedback}
A negative exponent means reciprocal: $p^{-3} = \dfrac{1}{p^3}$.
\end{feedback}
\end{problem}

\begin{problem}
Simplify $4^0 + 7^0$. $\answer{2}$
\begin{feedback}
Both terms equal $1$: $4^0 + 7^0 = 1 + 1 = 2$.
\end{feedback}
\end{problem}

\begin{problem}
Simplify $\dfrac{x^5 y^3}{x^2 y}$. $\answer{x^3 y^2}$
\begin{feedback}
Subtract exponents for each base: $x^{5-2} y^{3-1} = x^3 y^2$.
\end{feedback}
\end{problem}

% filepath: /code/ModuleThree/exponentsandscientificnotation.tex
% ...existing code...

\begin{problem}
Simplify $\dfrac{2x^4 y^{-2}}{8x^{-1} y^3}$. $\answer{\dfrac{x^5}{4y^5}}$
\begin{feedback}
Divide coefficients: $\frac{2}{8} = \frac{1}{4}$. For $x$: $x^{4-(-1)} = x^5$. For $y$: $y^{-2-3} = y^{-5} = \frac{1}{y^5}$. Final answer: $\frac{x^5}{4y^5}$.
\end{feedback}
\end{problem}

\begin{problem}
Simplify $(3a^2 b^{-3})^4$. $\answer{81a^8 b^{-12}}$
\begin{feedback}
Raise $3$ to the $4$th power: $3^4 = 81$. Multiply exponents: $a^{2 \times 4} = a^8$, $b^{-3 \times 4} = b^{-12}$. Final answer: $81a^8 b^{-12}$.
\end{feedback}
\end{problem}

\begin{problem}
Simplify $\dfrac{(x^2 y^{-1})^3}{x^{-4} y^2}$. $\answer{x^{10} y^{-5}}$
\begin{feedback}
Numerator: $(x^2)^3 = x^6$, $(y^{-1})^3 = y^{-3}$, so $x^6 y^{-3}$. Denominator: $x^{-4} y^2$. Subtract exponents: $x^{6-(-4)} = x^{10}$, $y^{-3-2} = y^{-5}$. Final answer: $x^{10} y^{-5}$.
\end{feedback}
\end{problem}

\begin{problem}
Simplify $(2x^{-2} y^3)^0$. $\answer{1}$
\begin{feedback}
Any nonzero expression to the zero power is $1$, regardless of the variables inside.
\end{feedback}
\end{problem}

\begin{problem}
Simplify $\dfrac{(a^3 b^{-2})^2}{a^{-1} b^4}$. $\answer{a^7 b^{-8}}$
\begin{feedback}
Numerator: $a^{3 \times 2} = a^6$, $b^{-2 \times 2} = b^{-4}$. Denominator: $a^{-1} b^4$. Subtract exponents: $a^{6-(-1)} = a^7$, $b^{-4-4} = b^{-8}$. Final answer: $a^7 b^{-8}$.
\end{feedback}
\end{problem}

% ...existing code...

\section*{Scientific Notation}

Scientific Notation (ERAU|10:16)

\youtube{SARy1UL-JPI}

% filepath: /code/ModuleThree/exponentsandscientificnotation.tex
% ...existing code...

\section*{Converting Scientific Notation to Decimal Notation}

\begin{problem}
Convert $3.2 \times 10^4$ to decimal notation. $\answer{32000}$
\begin{feedback}
Move the decimal point $4$ places to the right: $3.2 \rightarrow 32000$.
\end{feedback}
\end{problem}

\begin{problem}
Convert $7.05 \times 10^{-3}$ to decimal notation. $\answer{0.00705}$
\begin{feedback}
Move the decimal point $3$ places to the left: $7.05 \rightarrow 0.00705$.
\end{feedback}
\end{problem}

\begin{problem}
Convert $1.8 \times 10^2$ to decimal notation. $\answer{180}$
\begin{feedback}
Move the decimal point $2$ places to the right: $1.8 \rightarrow 180$.
\end{feedback}
\end{problem}

\begin{problem}
Convert $4.56 \times 10^{-1}$ to decimal notation. $\answer{0.456}$
\begin{feedback}
Move the decimal point $1$ place to the left: $4.56 \rightarrow 0.456$.
\end{feedback}
\end{problem}

% ...existing code...

% filepath: /code/ModuleThree/exponentsandscientificnotation.tex
% ...existing code...

\section*{Converting Decimal Notation to Scientific Notation}

\begin{problem}
Convert $45000$ to scientific notation. $\answer{4.5 \times 10^4}$
\begin{feedback}
Move the decimal point $4$ places to the left: $45000 \rightarrow 4.5 \times 10^4$.
\end{feedback}
\end{problem}

\begin{problem}
Convert $0.0083$ to scientific notation. $\answer{8.3 \times 10^{-3}}$
\begin{feedback}
Move the decimal point $3$ places to the right to get $8.3$, so the exponent is $-3$: $0.0083 \rightarrow 8.3 \times 10^{-3}$.
\end{feedback}
\end{problem}

\begin{problem}
Convert $123.6$ to scientific notation. $\answer{1.236 \times 10^2}$
\begin{feedback}
Move the decimal point $2$ places to the left: $123.6 \rightarrow 1.236 \times 10^2$.
\end{feedback}
\end{problem}

\begin{problem}
Convert $0.47$ to scientific notation. $\answer{4.7 \times 10^{-1}}$
\begin{feedback}
Move the decimal point $1$ place to the right to get $4.7$, so the exponent is $-1$: $0.47 \rightarrow 4.7 \times 10^{-1}$.
\end{feedback}
\end{problem}

% ...existing code...

% filepath: /code/ModuleThree/exponentsandscientificnotation.tex
% ...existing code...

\section*{Multiplying and Dividing in Scientific Notation}

\begin{problem}
Multiply $(2 \times 10^3) \times (5 \times 10^4)$. $\answer{1 \times 10^8}$
\begin{feedback}
Multiply the coefficients: $2 \times 5 = 10$. Add the exponents: $10^{3+4} = 10^7$. Write $10 \times 10^7 = 1 \times 10^8$.
\end{feedback}
\end{problem}

\begin{problem}
Divide $\dfrac{6 \times 10^5}{3 \times 10^2}$. $\answer{2 \times 10^3}$
\begin{feedback}
Divide the coefficients: $6 \div 3 = 2$. Subtract the exponents: $10^{5-2} = 10^3$. Final answer: $2 \times 10^3$.
\end{feedback}
\end{problem}

\begin{problem}
Multiply $(4 \times 10^{-2}) \times (3 \times 10^6)$. $\answer{1.2 \times 10^5}$
\begin{feedback}
Multiply the coefficients: $4 \times 3 = 12$. Add the exponents: $10^{-2+6} = 10^4$. Write $12 \times 10^4 = 1.2 \times 10^5$.
\end{feedback}
\end{problem}

\begin{problem}
Divide $\dfrac{8 \times 10^{-3}}{2 \times 10^{-6}}$. $\answer{4 \times 10^3}$
\begin{feedback}
Divide the coefficients: $8 \div 2 = 4$. Subtract the exponents: $-3 - (-6) = 3$. Final answer: $4 \times 10^3$.
\end{feedback}
\end{problem}

% ...existing code...

\end{document}
