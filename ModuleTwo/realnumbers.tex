\documentclass{ximera}

\title{Real Numbers}
\author{Amy Riordan}

\begin{document}
\begin{abstract}
Real Numbers video resources from Module 2.
\end{abstract}
\maketitle

\section*{Real Numbers}

Below each video, you’ll find a set of practice problems. Try those problems first. If you run into any difficulties, go back and rewatch the video, then attempt the problems again. This way, you’ll reinforce what you’ve learned and build a stronger understanding.

\section*{Real Numbers}

Real Numbers (ERAU|4:03)

% filepath: /code/realnumbers.tex
\section*{Identifying Types of Numbers}

For each number below, identify its type(s): whole number, natural number, integer, rational number, or irrational number.

\begin{problem}
$7$

\begin{feedback}
$7$ is a natural number, whole number, integer, and rational number. It is not irrational.
\end{feedback}
\end{problem}

\begin{problem}
$-3$

\begin{feedback}
$-3$ is an integer and a rational number. It is not a natural or whole number, and not irrational.
\end{feedback}
\end{problem}

\begin{problem}
$0$

\begin{feedback}
$0$ is a whole number, integer, and rational number. It is not a natural number or irrational.
\end{feedback}
\end{problem}

\begin{problem}
$\sqrt{5}$

\begin{feedback}
$\sqrt{5}$ is an irrational number because it cannot be written as a fraction and its decimal form does not terminate or repeat.
\end{feedback}
\end{problem}

\begin{problem}
$\frac{8}{3}$

\begin{feedback}
$\frac{8}{3}$ is a rational number because it can be written as a fraction. It is not a whole number, natural number, or integer.
\end{feedback}
\end{problem}

\section*{Real Numbers Properties}

Real Numbers Properties (ERAU|5:35)

% filepath: /code/realnumbers.tex
\section*{Identifying Properties of Real Numbers}

For each problem below, identify the property illustrated and read the feedback for clarification.

\begin{problem}
$3 + 0 = 3$

\begin{feedback}
This is the **Identity Property of Addition**. Adding zero to any number does not change its value.
\end{feedback}
\end{problem}

\begin{problem}
$5 \times 1 = 5$

\begin{feedback}
This is the **Identity Property of Multiplication**. Multiplying any number by one does not change its value.
\end{feedback}
\end{problem}

\begin{problem}
$4 + 7 = 7 + 4$

\begin{feedback}
This is the **Commutative Property of Addition**. The order in which numbers are added does not affect the sum.
\end{feedback}
\end{problem}

\begin{problem}
$(2 \times 3) \times 5 = 2 \times (3 \times 5)$

\begin{feedback}
This is the **Associative Property of Multiplication**. The way numbers are grouped when multiplying does not affect the product.
\end{feedback}
\end{problem}

\begin{problem}
$6 \times (4 + 2) = 6 \times 4 + 6 \times 2$

\begin{feedback}
This is the **Distributive Property**. Multiplying a number by a sum is the same as multiplying each addend by the number and then adding the results.
\end{feedback}
\end{problem}

\end{document}
