\documentclass{ximera}

\title{Linear Equations}
\author{Amy Riordan}

\begin{document}
\begin{abstract}
Linear Equations video resources from Module 2.
\end{abstract}
\maketitle

\section*{Linear Equations}

Below each video, you’ll find a set of practice problems. Try those problems first. If you run into any difficulties, go back and rewatch the video, then attempt the problems again. This way, you’ll reinforce what you’ve learned and build a stronger understanding.

\section*{Linear Equations}

Linear Equations in One Variable (Part 1) (ERAU|6:49)

\youtube{placeholder}

Linear Equations in One Variable (Part 2) (ERAU|6:45)

\youtube{placeholder}

% filepath: /code/ModuleTwo/linearequations.tex
% ...existing code...

\section*{Solving Linear Equations}

\begin{problem}
$2x + 5 = 13,\quad x = \answer{4}$
\begin{feedback}
Subtract $5$ from both sides: $2x = 8$. Divide both sides by $2$: $x = 4$.
\end{feedback}
\end{problem}

\begin{problem}
$3y - 7 = 2,\quad y = \answer{3}$
\begin{feedback}
Add $7$ to both sides: $3y = 9$. Divide both sides by $3$: $y = 3$.
\end{feedback}
\end{problem}

\begin{problem}
$5z = 20,\quad z = \answer{4}$
\begin{feedback}
Divide both sides by $5$: $z = 4$.
\end{feedback}
\end{problem}

\begin{problem}
$4x + 2 = 3x + 10,\quad x = \answer{8}$
\begin{feedback}
Subtract $3x$ from both sides: $x + 2 = 10$. Subtract $2$ from both sides: $x = 8$.
\end{feedback}
\end{problem}

\begin{problem}
$7y - 4 = 3y + 12,\quad y = \answer{4}$
\begin{feedback}
Subtract $3y$ from both sides: $4y - 4 = 12$. Add $4$ to both sides: $4y = 16$. Divide both sides by $4$: $y = 4$.
\end{feedback}
\end{problem}

\begin{problem}
$\frac{x}{3} + 2 = 5,\quad x = \answer{9}$
\begin{feedback}
Subtract $2$ from both sides: $\frac{x}{3} = 3$. Multiply both sides by $3$: $x = 9$.
\end{feedback}
\end{problem}

\begin{problem}
$\frac{2y}{5} - 4 = 6,\quad y = \answer{25}$
\begin{feedback}
Add $4$ to both sides: $\frac{2y}{5} = 10$. Multiply both sides by $5$: $2y = 50$. Divide both sides by $2$: $y = 25$.
\end{feedback}
\end{problem}

\begin{problem}
$\frac{z+1}{4} = 3,\quad z = \answer{11}$
\begin{feedback}
Multiply both sides by $4$: $z+1 = 12$. Subtract $1$ from both sides: $z = 11$.
\end{feedback}
\end{problem}

\begin{problem}
$2(x + 3) = 14,\quad x = \answer{4}$
\begin{feedback}
First, divide both sides by $2$: $x + 3 = 7$. Then subtract $3$ from both sides: $x = 4$.
\end{feedback}
\end{problem}

\begin{problem}
$2(x + 1) + 3(x - 2) = 13,\quad x = \answer{3}$
\begin{feedback}
Apply the distributive property: $2x + 2 + 3x - 6 = 13$. Combine like terms: $5x - 4 = 13$. Add $4$ to both sides: $5x = 17$. Divide both sides by $5$: $x = 3.4$.
\end{feedback}
\end{problem}

\begin{problem}
$5(y - 2) + 3 = 2(y + 4) + 7,\quad y = \answer{5}$
\begin{feedback}
Expand both sides: $5y - 10 + 3 = 2y + 8 + 7$. Simplify: $5y - 7 = 2y + 15$. Subtract $2y$ from both sides: $3y - 7 = 15$. Add $7$: $3y = 22$. Divide by $3$: $y = \frac{22}{3}$.
\end{feedback}
\end{problem}

\begin{problem}
$3(z + 2) + 2(z - 1) = 4(z - 3) + 10,\quad z = \answer{4}$
\begin{feedback}
Expand both sides: $3z + 6 + 2z - 2 = 4z - 12 + 10$. Simplify: $5z + 4 = 4z - 2$. Subtract $4z$ from both sides: $z + 4 = -2$. Subtract $4$: $z = -6$.
\end{feedback}
\end{problem}

% ...existing code...

\section*{Applications of Linear Equations}

Applications of Linear Equations in One Variable (Part 1) (ERAU|4:18)

\youtube{placeholder}

Applications of Linear Equations in One Variable (Part 2) (ERAU|8:30)

\youtube{placeholder}

% filepath: /code/ModuleTwo/linearequations.tex
% ...existing code...

\section*{Applications of Linear Equations Practice}

\begin{problem}
Sarah bought 3 notebooks and a pen for \$11. Each notebook costs \$3, and the pen costs \$x. What is the cost of the pen? $x=\answer{2}$
\begin{feedback}
Set up the equation: $3 \times 3 + x = 11 \implies 9 + x = 11$. Subtract $9$ from both sides: $x = 2$.
\end{feedback}
\end{problem}

\begin{problem}
A movie ticket and a popcorn cost \$15 together. The ticket costs \$x and the popcorn costs \$7. What is the price of the ticket? $x=\answer{8}$
\begin{feedback}
Set up the equation: $x + 7 = 15$. Subtract $7$ from both sides: $x = 8$.
\end{feedback}
\end{problem}

\begin{problem}
A rectangle has a perimeter of 30 units. Its length is $x$ units and its width is 5 units. Find the length. $x=\answer{10}$
\begin{feedback}
Perimeter formula: $2x + 2 \times 5 = 30 \implies 2x + 10 = 30$. Subtract $10$: $2x = 20$. Divide by $2$: $x = 10$.
\end{feedback}
\end{problem}

\begin{problem}
John worked 5 hours at \$12 per hour and earned a total of \$x. How much did he earn? $x=\answer{60}$
\begin{feedback}
Set up the equation: $5 \times 12 = x \implies x = 60$.
\end{feedback}
\end{problem}

\begin{problem}
A number increased by 7 is equal to 20. What is the number? $x=\answer{13}$
\begin{feedback}
Set up the equation: $x + 7 = 20$. Subtract $7$: $x = 13$.
\end{feedback}
\end{problem}

% ...existing code...

% filepath: /code/ModuleTwo/linearequations.tex
% ...existing code...

\begin{problem}
A taxi charges a \$4 base fare plus \$2.50 per mile. If the total fare was \$21, how many miles did the passenger travel? $x=\answer{6.8}$
\begin{feedback}
Set up the equation: $4 + 2.5x = 21$. Subtract $4$: $2.5x = 17$. Divide by $2.5$: $x = 6.8$ miles.
\end{feedback}
\end{problem}

\begin{problem}
A cell phone plan costs \$30 per month plus \$0.10 for each text message sent. If the bill for one month was \$45, how many text messages were sent? $x=\answer{150}$
\begin{feedback}
Set up the equation: $30 + 0.10x = 45$. Subtract $30$: $0.10x = 15$. Divide by $0.10$: $x = 150$ messages.
\end{feedback}
\end{problem}

\begin{problem}
The sum of three consecutive integers is 51. What are the integers? Let the smallest integer be $x$. $x=\answer{16}$
\begin{feedback}
Set up the equation: $x + (x+1) + (x+2) = 51 \implies 3x + 3 = 51$. Subtract $3$: $3x = 48$. Divide by $3$: $x = 16$. The integers are $16$, $17$, and $18$.
\end{feedback}
\end{problem}

% ...existing code...

\end{document}
