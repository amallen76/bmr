\documentclass{ximera}

\title{Linear Equations}
\author{Amy Riordan}

\begin{document}
\begin{abstract}
Linear Equations video resources from Module 2.
\end{abstract}
\maketitle

\section*{Linear Equations}

Below each video, you’ll find a set of practice problems. Try those problems first. If you run into any difficulties, go back and rewatch the video, then attempt the problems again. This way, you’ll reinforce what you’ve learned and build a stronger understanding.

\section*{Linear Equations}

Linear Equations in One Variable (Part 1) (ERAU|6:49)

\youtube{placeholder}

% filepath: /code/ModuleTwo/linearequations.tex

\section*{Practice Problems: Solving Linear Equations}

\begin{problem}
$2x + 5 = 13,\quad x = \answer{4}$
\begin{feedback}
Subtract $5$ from both sides: $2x = 8$. Divide both sides by $2$: $x = 4$.
\end{feedback}
\end{problem}

\begin{problem}
$3y - 7 = 2,\quad y = \answer{3}$
\begin{feedback}
Add $7$ to both sides: $3y = 9$. Divide both sides by $3$: $y = 3$.
\end{feedback}
\end{problem}

\begin{problem}
$5z = 20,\quad z = \answer{4}$
\begin{feedback}
Divide both sides by $5$: $z = 4$.
\end{feedback}
\end{problem}

\begin{problem}
$4x + 2 = 3x + 10,\quad x = \answer{8}$
\begin{feedback}
Subtract $3x$ from both sides: $x + 2 = 10$. Subtract $2$ from both sides: $x = 8$.
\end{feedback}
\end{problem}

\begin{problem}
$7y - 4 = 3y + 12,\quad y = \answer{4}$
\begin{feedback}
Subtract $3y$ from both sides: $4y - 4 = 12$. Add $4$ to both sides: $4y = 16$. Divide both sides by $4$: $y = 4$.
\end{feedback}
\end{problem}

Linear Equations in One Variable (Part 2) (ERAU|6:45)

\youtube{placeholder}

% filepath: /code/ModuleTwo/linearequations.tex
% ...existing code...

\begin{problem}
$\frac{x}{3} + 2 = 5,\quad x = \answer{9}$
\begin{feedback}
Subtract $2$ from both sides: $\frac{x}{3} = 3$. Multiply both sides by $3$: $x = 9$.
\end{feedback}
\end{problem}

\begin{problem}
$\frac{2y}{5} - 4 = 6,\quad y = \answer{25}$
\begin{feedback}
Add $4$ to both sides: $\frac{2y}{5} = 10$. Multiply both sides by $5$: $2y = 50$. Divide both sides by $2$: $y = 25$.
\end{feedback}
\end{problem}

\begin{problem}
$\frac{z+1}{4} = 3,\quad z = \answer{11}$
\begin{feedback}
Multiply both sides by $4$: $z+1 = 12$. Subtract $1$ from both sides: $z = 11$.
\end{feedback}
\end{problem}

% ...existing code...

% filepath: /code/ModuleTwo/linearequations.tex
% ...existing code...

\begin{problem}
$2(x + 3) = 14,\quad x = \answer{4}$
\begin{feedback}
First, divide both sides by $2$: $x + 3 = 7$. Then subtract $3$ from both sides: $x = 4$.
\end{feedback}
\end{problem}

\begin{problem}
$2(x + 1) + 3(x - 2) = 13,\quad x = \answer{3}$
\begin{feedback}
Apply the distributive property: $2x + 2 + 3x - 6 = 13$. Combine like terms: $5x - 4 = 13$. Add $4$ to both sides: $5x = 17$. Divide both sides by $5$: $x = 3.4$.
\end{feedback}
\end{problem}

\begin{problem}
$5(y - 2) + 3 = 2(y + 4) + 7,\quad y = \answer{5}$
\begin{feedback}
Expand both sides: $5y - 10 + 3 = 2y + 8 + 7$. Simplify: $5y - 7 = 2y + 15$. Subtract $2y$ from both sides: $3y - 7 = 15$. Add $7$: $3y = 22$. Divide by $3$: $y = \frac{22}{3}$.
\end{feedback}
\end{problem}

\begin{problem}
$3(z + 2) + 2(z - 1) = 4(z - 3) + 10,\quad z = \answer{4}$
\begin{feedback}
Expand both sides: $3z + 6 + 2z - 2 = 4z - 12 + 10$. Simplify: $5z + 4 = 4z - 2$. Subtract $4z$ from both sides: $z + 4 = -2$. Subtract $4$: $z = -6$.
\end{feedback}
\end{problem}

% ...existing code...

\end{document}
