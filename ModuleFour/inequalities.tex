\documentclass{ximera}

\title{Inequalities}
\author{Amy Riordan}

\begin{document}
\begin{abstract}
Inequalities video resources from Module 4.
\end{abstract}
\maketitle

\section*{Inequalities}

Below each video, you’ll find a set of practice problems. Try those problems first. If you run into any difficulties, go back and rewatch the video, then attempt the problems again. This way, you’ll reinforce what you’ve learned and build a stronger understanding.

\section*{Linear Inequalities in Two Variables}

Linear Inequalities in Two Variables (ERAU|7:57)

\youtube{placeholder}

% ...existing code...

\section*{Graphing Linear Inequalities in Two Variables}

\begin{problem}
Graph the inequality \(y > -2x + 3\). Describe the boundary and the solution region. Is the point \((0,0)\) a solution?
$\answer{\text{Boundary: dashed }y=-2x+3;\ \text{Shade above the line};\ (0,0)\ \text{is not a solution}}$
\begin{feedback}
Rewrite as \(y > -2x+3\). The boundary is the line \(y=-2x+3\) and is dashed because the inequality is strict. Test the point \((0,0)\): \(0 > 3\) is false, so \((0,0)\) is not in the solution region. Shade the region of points with \(y\) greater than \(-2x+3\).
\end{feedback}
\end{problem}

\begin{problem}
Graph the inequality \(2x + y \le 4\). Describe the boundary and the solution region. Is the point \((0,0)\) a solution?
$\answer{\text{Boundary: solid }y=4-2x;\ \text{Shade below or on the line};\ (0,0)\ \text{is a solution}}$
\begin{feedback}
Solve for \(y\): \(y \le 4-2x\). The boundary is the line \(y=4-2x\) and is solid because \(\le\) includes the boundary. Test \((0,0)\): \(2(0)+0 \le 4\) gives \(0\le4\), true. Shade the region at or below the line.
\end{feedback}
\end{problem}

\begin{problem}
Graph the inequality \(x - 3y \ge 6\). Describe the boundary and the solution region. Is the point \((6,0)\) a solution?
$\answer{\text{Boundary: solid }y=\dfrac{x-6}{3};\ \text{Shade below or on the line};\ (6,0)\ \text{is a solution}}$
\begin{feedback}
Rewrite: \(x-3y\ge6 \Rightarrow -3y\ge6-x \Rightarrow 3y\le x-6 \Rightarrow y\le\dfrac{x-6}{3}\). The boundary is \(y=(x-6)/3\) and is solid (``\(\ge\)'' includes boundary). Test \((6,0)\): \(6-3(0)=6\ge6\) is true, so \((6,0)\) lies on the boundary and is included. Shade the region at or below the line.
\end{feedback}
\end{problem}

% ...existing code...

% ...existing code...

\section*{Graphing Linear Inequalities in One Variable}

\begin{problem}
Graph the inequality \(x > 3\). Describe the boundary and the solution region. Is the point \(3\) a solution?
$\answer{\text{Boundary: open circle at }3;\ \text{Shade to the right};\ 3\ \text{is not a solution}}$
\begin{feedback}
The boundary is the point \(x=3\) shown with an open circle because the inequality is strict. The solution set is all real numbers greater than 3 (shade to the right). Test \(x=4\): \(4>3\) is true. Test \(x=3\): \(3>3\) is false, so \(3\) is not included.
\end{feedback}
\end{problem}

\begin{problem}
Graph the compound inequality \(-2 \le x \le 4\). Describe the boundary and the solution region. Is the point \(0\) a solution?
$\answer{\text{Boundary: closed at }-2\text{ and }4;\ \text{Shade between and including endpoints};\ 0\ \text{is a solution}}$
\begin{feedback}
Both endpoints are included, so use solid dots at \(-2\) and \(4\). The solution set is all real numbers between \(-2\) and \(4\), inclusive. Test \(x=0\): \(-2\le0\le4\) is true, so \(0\) lies in the solution set.
\end{feedback}
\end{problem}

% ...existing code...

\section*{Systems of Linear Inequalities in Two Variables}

Systems of Linear Inequalities in Two Variables (ERAU|6:45)

\youtube{placeholder}

% ...existing code...

\section*{Graphing Systems of Linear Inequalities in Two Variables}

\begin{problem}
Graph the system and describe the solution region. Give one test point that is in the solution.
\[
\begin{cases}
y > x - 1\\[4pt]
y < -x + 3
\end{cases}
\]
$\answer{\text{Region strictly between the two dashed lines; e.g. }(1,1)}$
\begin{feedback}
Boundaries: $y=x-1$ and $y=-x+3$, both dashed because the inequalities are strict. Shade the region above $y=x-1$ and below $y=-x+3$. Test $(1,1)$: $1>1-1=0$ (true) and $1< -1+3=2$ (true), so $(1,1)$ lies in the solution region.
\end{feedback}
\end{problem}

\begin{problem}
Graph the system and describe the solution region. Give one test point that is in the solution.
\[
\begin{cases}
y \ge 2x - 2\\[4pt]
y \le \tfrac{1}{2}x + 4
\end{cases}
\]
$\answer{\text{Region between the two lines including boundaries; e.g. }(2,3)}$
\begin{feedback}
Boundaries: $y=2x-2$ (solid) and $y=\tfrac{1}{2}x+4$ (solid). Shade the region on or above $y=2x-2$ and on or below $y=\tfrac{1}{2}x+4$. Test $(2,3)$: $3\ge2(2)-2=2$ (true) and $3\le\tfrac{1}{2}(2)+4=5$ (true), so $(2,3)$ is in the solution region.
\end{feedback}
\end{problem}

\begin{problem}
Graph the system and determine whether any solution exists. If none, explain why.
\[
\begin{cases}
y > x + 1\\[4pt]
y \le x - 1
\end{cases}
\]
$\answer{No\ solution}$
\begin{feedback}
The first inequality requires $y-x>1$, while the second requires $y-x\le-1$. These conditions cannot both hold for the same point, so the solution set is empty (no region to shade).
\end{feedback}
\end{problem}

% ...existing code...

% ...existing code...

\section*{Systems of Linear Inequalities in One Variable}

\begin{problem}
Solve the system by graphing:
\[
\begin{cases}
x > -1\\[4pt]
x \le 4
\end{cases}
\]
$\answer{(-1,4]}$
\begin{feedback}
Intersection: all $x$ greater than $-1$ and at most $4$. Boundary at $-1$ is open, at $4$ is closed. Solution set is $(-1,4]$. Test $x=0$: $0>-1$ and $0\le4$ (both true).
\end{feedback}
\end{problem}

\begin{problem}
Solve the system by graphing:
\[
\begin{cases}
x \le -2\\[4pt]
x < 3
\end{cases}
\]
$\answer{(-\infty,-2]}$
\begin{feedback}
Intersection: numbers at most $-2$ that are also less than $3$. The tighter restriction is $x\le-2$, so the solution is $(-\infty,-2]$. Boundary at $-2$ is closed. Test $x=-3$: $-3\le-2$ and $-3<3$ (both true).
\end{feedback}
\end{problem}

% ...existing code...

\end{document}
