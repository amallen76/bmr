\documentclass{ximera}

\title{Systems of Equations}
\author{Amy Riordan}

\begin{document}
\begin{abstract}
Systems of Equations video resources from Module 4.
\end{abstract}
\maketitle

\section*{Systems of Equations}

Below each video, you’ll find a set of practice problems. Try those problems first. If you run into any difficulties, go back and rewatch the video, then attempt the problems again. This way, you’ll reinforce what you’ve learned and build a stronger understanding.

\section*{Solving Systems of Linear Equations by Graphing}

Solving Systems of Linear Equations by Graphing (ERAU|3:37)

\youtube{kjsp6qLmp2U}

% ...existing code...
\section*{Determining whether a point is a solution}

Determine whether each given point is a solution of the system. Show your work.

\begin{problem}
Is the point $(2,1)$ a solution of
\[
\begin{cases}
2x + 3y = 7\\
x - y = 1
\end{cases}
\]
$\answer{Yes}$
\begin{feedback}
Substitute $x=2$, $y=1$: $2(2)+3(1)=4+3=7$ and $2-1=1$. Both equations are true, so $(2,1)$ is a solution.
\end{feedback}
\end{problem}

\begin{problem}
Is the point $(1,1)$ a solution of
\[
\begin{cases}
x + 4y = 5\\
-\,x + 2y = 1
\end{cases}
\]
$\answer{Yes}$
\begin{feedback}
Substitute $x=1$, $y=1$: $1+4(1)=5$ and $-1+2(1)=1$. Both equations are true, so $(1,1)$ is a solution.
\end{feedback}
\end{problem}

\begin{problem}
Is the point $(2,3)$ a solution of
\[
\begin{cases}
3x - y = 4\\
x + y = 5
\end{cases}
\]
$\answer{No}$
\begin{feedback}
Substitute $x=2$, $y=3$: $3(2)-3=6-3=3\neq4$, so the first equation is false (the second is true: $2+3=5$). Therefore $(2,3)$ is not a solution.
\end{feedback}
\end{problem}

% ...existing code...

\section*{Solving Systems of Linear Equations by Graphing}

\begin{problem}
Solve the system by graphing:
\[
\begin{cases}
x + y = 5\\
x - y = 1
\end{cases}
\]
$\answer{(3,2)}$
\begin{feedback}
Graph both lines: $y = -x + 5$ and $y = x - 1$. Their intersection is $(3,2)$. Check: $3+2=5$ and $3-2=1$, so $(3,2)$ is the solution.
\end{feedback}
\end{problem}

\begin{problem}
Solve the system by graphing:
\[
\begin{cases}
2x + y = 9\\
x - y = 3
\end{cases}
\]
$\answer{(4,1)}$
\begin{feedback}
Rewrite as $y = -2x + 9$ and $y = x - 3$. The lines intersect at $(4,1)$. Check: $2(4)+1=9$ and $4-1=3$, so $(4,1)$ is the solution.
\end{feedback}
\end{problem}

\begin{problem}
Solve the system by graphing:
\[
\begin{cases}
x + y = 7\\
2x - y = -1
\end{cases}
\]
$\answer{(2,5)}$
\begin{feedback}
Rewrite as $y = -x + 7$ and $y = 2x + 1$. The intersection point is $(2,5)$. Check: $2+5=7$ and $2(2)-5=-1$, so $(2,5)$ is the solution.
\end{feedback}
\end{problem}

% ...existing code...

\section*{Solving Systems of Linear Equations by Substitution}

Solving Systems of Linear Equations by Substitution (ERAU|3:57)

\youtube{iey5k8taAFY}

\section*{Solving Systems of Linear Equations by Substitution}

\begin{problem}
Solve the system using substitution:
\[
\begin{cases}
y = 2x + 1\\
3x - y = 4
\end{cases}
\]
$\answer{(5,11)}$
\begin{feedback}
Substitute $y=2x+1$ into $3x-y=4$: $3x-(2x+1)=4\Rightarrow x-1=4\Rightarrow x=5$. Then $y=2(5)+1=11$. Check: $3(5)-11=4$, so $(5,11)$ is the solution.
\end{feedback}
\end{problem}

\begin{problem}
Solve the system using substitution:
\[
\begin{cases}
x = 4 - y\\
2x + 3y = 10
\end{cases}
\]
$\answer{(2,2)}$
\begin{feedback}
Substitute $x=4-y$ into $2x+3y=10$: $2(4-y)+3y=10\Rightarrow 8-2y+3y=10\Rightarrow y=2$. Then $x=4-2=2$. Check: $2(2)+3(2)=10$, so $(2,2)$ is the solution.
\end{feedback}
\end{problem}

\begin{problem}
Solve the system using substitution:
\[
\begin{cases}
y = -x + 4\\
2x + y = 5
\end{cases}
\]
$\answer{(1,3)}$
\begin{feedback}
Substitute $y=-x+4$ into $2x+y=5$: $2x+(-x+4)=5\Rightarrow x+4=5\Rightarrow x=1$. Then $y=-1+4=3$. Check: $2(1)+3=5$, so $(1,3)$ is the solution.
\end{feedback}
\end{problem}

% ...existing code...

\section*{Solving Systems of Linear Equations by Addition}

Solving Systems of Linear Equations by Addition (ERAU|4:34)

\youtube{kgpNzCOjXzc}

\section*{Solving Systems of Linear Equations by Addition}

\begin{problem}
Solve the system using addition (elimination):
\[
\begin{cases}
x + y = 5\\
2x - y = 1
\end{cases}
\]
$\answer{(2,3)}$
\begin{feedback}
Add the equations to eliminate $y$: $(x+y)+(2x-y)=3x=6\Rightarrow x=2$. Substitute $x=2$ into $x+y=5$: $2+y=5\Rightarrow y=3$. Check: $2+3=5$ and $2(2)-3=1$.
\end{feedback}
\end{problem}

\begin{problem}
Solve the system using addition (elimination):
\[
\begin{cases}
2x + 4y = 10\\
3x - 4y = 5
\end{cases}
\]
$\answer{(3,1)}$
\begin{feedback}
Add the equations to eliminate $y$: $(2x+4y)+(3x-4y)=5x=15\Rightarrow x=3$. Substitute $x=3$ into $2x+4y=10$: $6+4y=10\Rightarrow 4y=4\Rightarrow y=1$. Check: $2(3)+4(1)=10$ and $3(3)-4(1)=5$.
\end{feedback}
\end{problem}

\begin{problem}
Solve the system using addition (elimination):
\[
\begin{cases}
3x + 2y = 16\\
-3x + y = -1
\end{cases}
\]
$\answer{(2,5)}$
\begin{feedback}
Add the equations to eliminate $x$: $(3x+2y)+(-3x+y)=3y=15\Rightarrow y=5$. Substitute $y=5$ into $3x+2y=16$: $3x+10=16\Rightarrow 3x=6\Rightarrow x=2$. Check: $3(2)+2(5)=16$ and $-3(2)+5=-1$.
\end{feedback}
\end{problem}

% ...existing code...

\section*{Systems of Linear Equations in Two Variables}

Systems of Linear Equations in Two Variables (ERAU|9:34)

\youtube{pernSj3-YOU}

% ...existing code...

\section*{Types of Solutions: One, None, and Infinitely Many}

\begin{problem}
Solve the system:
\[
\begin{cases}
x + y = 4\\
x - y = 2
\end{cases}
\]
$\answer{(3,1)}$
\begin{feedback}
Add the equations: $2x = 6 \Rightarrow x = 3$. Then $y = 4 - 3 = 1$. Check: $3+1=4$ and $3-1=2$. You have one solution at $(3,1)$.
\end{feedback}
\end{problem}

\begin{problem}
Solve the system:
\[
\begin{cases}
x + y = 3\\
2x + 2y = 8
\end{cases}
\]
$\answer{No\ solution}$
\begin{feedback}
From $x+y=3$ we get $y=3-x$. Substitute: $2x+2(3-x)=6$, which contradicts $8$. The lines are parallel and distinct, so there is no solution.
\end{feedback}
\end{problem}

\begin{problem}
Solve the system:
\[
\begin{cases}
x + y = 3\\
2x + 2y = 6
\end{cases}
\]
$\answer{Infinitely\ many\ solutions}$
\begin{feedback}
The second equation is $2$ times the first, so they represent the same line. Substitution yields an identity ($6=6$). There are infinitely many solutions; for example $y=3-x$ describes them.
\end{feedback}
\end{problem}

% ...existing code...

% ...existing code...

\section*{Word Problems: Systems of Equations}

\begin{problem}
A teacher bought 40 pens. Ballpoint pens cost \$3 each and gel pens cost \$5 each. She spent a total of \$160. How many of each type did she buy?
$\answer{(20\ \text{ballpoint},\ 20\ \text{gel})}$
\begin{feedback}
Let \(b\) = number of ballpoint pens, \(g\) = number of gel pens. Then
\[
\begin{cases}
b + g = 40\\[4pt]
3b + 5g = 160
\end{cases}
\]
Multiply the first equation by 3: \(3b+3g=120\). Subtract from the second: \((3b+5g)-(3b+3g)=2g=40\Rightarrow g=20\). Then \(b=40-20=20\). Check: \(3(20)+5(20)=60+100=160\).
\end{feedback}
\end{problem}

\begin{problem}
The perimeter of a rectangle is 50 meters. The length is 4 meters more than twice the width. Find the length and width.
$\answer{(18,\ 7)}$
\begin{feedback}
Let \(L\) = length, \(W\) = width. Then
\[
\begin{cases}
2L+2W=50\\[4pt]
L=2W+4
\end{cases}
\]
Substitute \(L\): \(2(2W+4)+2W=50 \Rightarrow 4W+8+2W=50 \Rightarrow 6W=42 \Rightarrow W=7\). Then \(L=2(7)+4=18\). Check: \(2(18)+2(7)=36+14=50\).
\end{feedback}
\end{problem}

\begin{problem}
A theater sold 60 tickets to a show. Adult tickets cost \$12 and child tickets cost \$7. The total ticket sales were \$660. How many adult and child tickets were sold?
$\answer{(48\ \text{adult},\ 12\ \text{child})}$
\begin{feedback}
Let \(a\) = adult tickets, \(c\) = child tickets. Then
\[
\begin{cases}
a+c=60\\[4pt]
12a+7c=660
\end{cases}
\]
Multiply the first equation by 7: \(7a+7c=420\). Subtract from the revenue equation: \((12a+7c)-(7a+7c)=5a=240\Rightarrow a=48\). Then \(c=60-48=12\). Check: \(12(48)+7(12)=576+84=660\).
\end{feedback}
\end{problem}

% ...existing code...


\end{document}
