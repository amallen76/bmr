\documentclass{ximera}

\title{Graphing}
\author{Amy Riordan}

\begin{document}
\begin{abstract}
Graphing video resources from Module 4.
\end{abstract}
\maketitle

\section*{Graphing Linear Equations}

Below each video, you’ll find a set of practice problems. Try those problems first. If you run into any difficulties, go back and rewatch the video, then attempt the problems again. This way, you’ll reinforce what you’ve learned and build a stronger understanding.

\section*{Linear Functions and Their Graphs}

Linear Functions and Their Graphs (ERAU|13:17)

\youtube{eoic_OoOwVg}

% filepath: /code/ModuleFour/graphing.tex
% ...existing code...

\section*{Graphing Lines Using Intercepts}

\begin{problem}
Graph the line $2x + 3y = 6$ using its intercepts. $\answer{2x + 3y = 6}$
\begin{feedback}
Find the $x$-intercept by setting $y = 0$: $2x = 6 \implies x = 3$. Find the $y$-intercept by setting $x = 0$: $3y = 6 \implies y = 2$. Plot $(3, 0)$ and $(0, 2)$, then draw a line through these points.
\end{feedback}
\end{problem}

\begin{problem}
Graph the line $x - 4y = 8$ using its intercepts. $\answer{x - 4y = 8}$
\begin{feedback}
Find the $x$-intercept by setting $y = 0$: $x = 8$. Find the $y$-intercept by setting $x = 0$: $-4y = 8 \implies y = -2$. Plot $(8, 0)$ and $(0, -2)$, then draw a line through these points.
\end{feedback}
\end{problem}

\begin{problem}
Graph the line $5x + y = 10$ using its intercepts. $\answer{5x + y = 10}$
\begin{feedback}
Find the $x$-intercept by setting $y = 0$: $5x = 10 \implies x = 2$. Find the $y$-intercept by setting $x = 0$: $y = 10$. Plot $(2, 0)$ and $(0, 10)$, then draw a line through these points.
\end{feedback}
\end{problem}

% ...existing code...

% filepath: /code/ModuleFour/graphing.tex
% ...existing code...

\section*{Finding the Slope Between Two Points}

\begin{problem}
Find the slope between the points $(2, 5)$ and $(6, 13)$. $\answer{2}$
\begin{feedback}
Use the formula $m = \frac{y_2 - y_1}{x_2 - x_1}$. Here, $m = \frac{13 - 5}{6 - 2} = \frac{8}{4} = 2$.
\end{feedback}
\end{problem}

\begin{problem}
Find the slope between the points $(-3, 4)$ and $(1, -8)$. $\answer{-3}$
\begin{feedback}
Use the formula $m = \frac{-8 - 4}{1 - (-3)} = \frac{-12}{4} = -3$.
\end{feedback}
\end{problem}

\begin{problem}
Find the slope between the points $(0, 0)$ and $(7, 2)$. $\answer{\frac{2}{7}}$
\begin{feedback}
Use the formula $m = \frac{2 - 0}{7 - 0} = \frac{2}{7}$.
\end{feedback}
\end{problem}

% ...existing code...

% filepath: /code/ModuleFour/graphing.tex
% ...existing code...

\section*{Graphing Lines Using Slope-Intercept Form}

\begin{problem}
Graph the line $y = 3x - 2$ using the slope-intercept form. $\answer{y = 3x - 2}$
\begin{feedback}
Start by plotting the y-intercept at $(0, -2)$. The slope is $3$, so from $(0, -2)$, go up $3$ units and right $1$ unit to $(1, 1)$. Draw a straight line through these points.
\end{feedback}
\end{problem}

\begin{problem}
Graph the line $y = -\frac{1}{2}x + 4$ using the slope-intercept form. $\answer{y = -\frac{1}{2}x + 4}$
\begin{feedback}
Plot the y-intercept at $(0, 4)$. The slope is $-\frac{1}{2}$, so from $(0, 4)$, go down $1$ unit and right $2$ units to $(2, 3)$. Draw a straight line through these points.
\end{feedback}
\end{problem}

\begin{problem}
Graph the line $y = x$ using the slope-intercept form. $\answer{y = x}$
\begin{feedback}
Plot the y-intercept at $(0, 0)$. The slope is $1$, so from $(0, 0)$, go up $1$ unit and right $1$ unit to $(1, 1)$. Draw a straight line through these points.
\end{feedback}
\end{problem}

% ...existing code...

% filepath: /code/ModuleFour/graphing.tex
% ...existing code...

\section*{Graphing Vertical and Horizontal Lines}

\begin{problem}
Graph the vertical line $x = -3$ on a coordinate plane. $\answer{x = -3}$
\begin{feedback}
A vertical line has the same $x$ value for all points. Plot points like $(-3, 0)$ and $(-3, 2)$, then draw a straight vertical line through $x = -3$.
\end{feedback}
\end{problem}

\begin{problem}
Graph the horizontal line $y = 5$ on a coordinate plane. $\answer{y = 5}$
\begin{feedback}
A horizontal line has the same $y$ value for all points. Plot points like $(0, 5)$ and $(3, 5)$, then draw a straight horizontal line through $y = 5$.
\end{feedback}
\end{problem}

% ...existing code...

\section*{The Point-Slope Form of a Line}

The Point-Slope Form of a Line (ERAU|6:24)

\youtube{a260NNVdUok}

% filepath: /code/ModuleFour/graphing.tex
% ...existing code...

\section*{Writing Point-Slope and Slope-Intercept Forms}

\begin{problem}
Write the equation of the line passing through $(2, 5)$ with slope $3$ in point-slope form and slope-intercept form. $\answer{y - 5 = 3(x - 2);\ y = 3x - 1}$
\begin{feedback}
Point-slope form: $y - 5 = 3(x - 2)$. Distribute and solve for $y$: $y = 3x - 6 + 5 = 3x - 1$.
\end{feedback}
\end{problem}

\begin{problem}
Write the equation of the line passing through $(-1, 4)$ with slope $-2$ in point-slope form and slope-intercept form. $\answer{y - 4 = -2(x + 1);\ y = -2x + 2}$
\begin{feedback}
Point-slope form: $y - 4 = -2(x + 1)$. Distribute and solve for $y$: $y = -2x - 2 + 4 = -2x + 2$.
\end{feedback}
\end{problem}

\begin{problem}
Write the equation of the line passing through $(0, -3)$ with slope $\frac{1}{2}$ in point-slope form and slope-intercept form. $\answer{y + 3 = \frac{1}{2}x;\ y = \frac{1}{2}x - 3}$
\begin{feedback}
Point-slope form: $y + 3 = \frac{1}{2}x$. Solve for $y$: $y = \frac{1}{2}x - 3$.
\end{feedback}
\end{problem}

\begin{problem}
Write the equation of the line passing through $(4, 0)$ with slope $-5$ in point-slope form and slope-intercept form. $\answer{y - 0 = -5(x - 4);\ y = -5x + 20}$
\begin{feedback}
Point-slope form: $y - 0 = -5(x - 4)$. Distribute and solve for $y$: $y = -5x + 20$.
\end{feedback}
\end{problem}

\begin{problem}
Write the equation of the line passing through $(3, -2)$ with slope $0$ in point-slope form and slope-intercept form. $\answer{y + 2 = 0(x - 3);\ y = -2}$
\begin{feedback}
Point-slope form: $y + 2 = 0(x - 3)$. Since the slope is $0$, the line is horizontal: $y = -2$.
\end{feedback}
\end{problem}

% ...existing code...

\section*{Writing Point-Slope and Slope-Intercept Forms from Two Points}

\begin{problem}
Write the equation of the line passing through $(1, 2)$ and $(4, 8)$ in point-slope form and slope-intercept form. $\answer{y - 2 = 2(x - 1);\ y = 2x}$ 
\begin{feedback}
First, find the slope: $m = \frac{8 - 2}{4 - 1} = \frac{6}{3} = 2$.  
Point-slope form: $y - 2 = 2(x - 1)$.  
Slope-intercept form: $y = 2x$ (since plugging in $(1, 2)$ gives $2 = 2(1)$).
\end{feedback}
\end{problem}

\begin{problem}
Write the equation of the line passing through $(-2, 5)$ and $(3, -5)$ in point-slope form and slope-intercept form. $\answer{y - 5 = -2(x + 2);\ y = -2x + 1}$
\begin{feedback}
Find the slope: $m = \frac{-5 - 5}{3 - (-2)} = \frac{-10}{5} = -2$.  
Point-slope form: $y - 5 = -2(x + 2)$.  
Slope-intercept form: $y = -2x + 1$ (using $y = mx + b$ and plugging in one point).
\end{feedback}
\end{problem}

\begin{problem}
Write the equation of the line passing through $(0, -3)$ and $(6, 9)$ in point-slope form and slope-intercept form. $\answer{y + 3 = 2(x - 0);\ y = 2x - 3}$
\begin{feedback}
Find the slope: $m = \frac{9 - (-3)}{6 - 0} = \frac{12}{6} = 2$.  
Point-slope form: $y + 3 = 2(x - 0)$ or $y + 3 = 2x$.  
Slope-intercept form: $y = 2x - 3$.
\end{feedback}
\end{problem}

% ...existing code...

\section*{Scatter Plots and Regression Lines}

Scatter Plots and Regression Lines (ERAU|3:51)

\youtube{WZ7VpYUt2eA}

% filepath: /code/ModuleFour/graphing.tex
% ...existing code...

\section*{Graphing Scatter Plots and Identifying Correlation}

\begin{problem}
Given the points $(1, 2)$, $(2, 4)$, $(3, 6)$, $(4, 8)$, graph the scatter plot and identify the correlation. $\answer{\text{Positive correlation}}$
\begin{feedback}
The points rise from left to right, showing a positive correlation. As $x$ increases, $y$ increases.
\end{feedback}
\end{problem}

\begin{problem}
Given the points $(1, 8)$, $(2, 6)$, $(3, 4)$, $(4, 2)$, graph the scatter plot and identify the correlation. $\answer{\text{Negative correlation}}$
\begin{feedback}
The points fall from left to right, showing a negative correlation. As $x$ increases, $y$ decreases.
\end{feedback}
\end{problem}

\begin{problem}
Given the points $(1, 5)$, $(2, 3)$, $(3, 7)$, $(4, 2)$, graph the scatter plot and identify the correlation. $\answer{\text{No correlation}}$
\begin{feedback}
The points do not show a clear pattern; there is no correlation between $x$ and $y$.
\end{feedback}
\end{problem}

% ...existing code...

\end{document}
