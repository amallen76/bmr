% filepath: /code/newActivityMod 1.tex
\documentclass{ximera}

\title{Integer Videos}
\author{Amy Riordan}

\begin{document}
\begin{abstract}
Integer videos resources from Module 1.
\end{abstract}
\maketitle

\section*{Integer Videos}

Below each video, you'll find a set of practice problems. Try those problems first. If you run into any difficulties, go back and rewatch the video, then attempt the problems again. This way, you'll reinforce what you’ve learned and build a stronger understanding.

\section*{Integers}

Integers (ERAU | 6:16)

\youtube{GubPk8_wInE}

\section*{Integers}

% filepath: /code/integers.tex
% ...existing code...

Insert either $<$ or $>$ in the space between the integers to make each statement true.

\begin{problem}
$8\_\_5$\\
$\answer{>}$

\begin{feedback}
Since $8$ is to the right of $5$ on the number line, $8 > 5$.
\end{feedback}
\end{problem}

\begin{problem}
$-3\_\_2$\\
$\answer{<}$

\begin{feedback}
Remember, negative numbers are always less than positive numbers. Since $-3$ is to the left of $2$ on the number line, $-3 < 2$.
\end{feedback}

\end{problem}

\begin{problem}
$0\_\_-7$\\
$\answer{>}$

\begin{feedback}
Since $0$ is to the right of $-7$ on the number line, $0 > -7$.
\end{feedback}

\end{problem}

% ...existing code...

% filepath: /code/integers.tex
% ...existing code...

Insert either $\leq$ or $\geq$ in the space to make the statement true.

\begin{problem}
$7\_\_7$\\
$\answer{\geq}$

\begin{feedback}
Since $7$ is equal to $7$, we can use the symbol $\geq$ (greater than or equal to) to make the statement true.
\end{feedback}

\end{problem}

\begin{problem}
$-4\_\_2$\\
$\answer{\leq}$

\begin{feedback}
Since $-4$ is less than $2$, we can use the symbol $\leq$ (less than or equal to) to make the statement true.
\end{feedback}

\end{problem}

\begin{problem}
$5\_\_9$\\
$\answer{\leq}$

\begin{feedback}
Since $5$ is less than $9$, we can use the symbol $\leq$ (less than or equal to) to make the statement true.
\end{feedback}

\end{problem}

% ...existing code...

% filepath: /code/integers.tex
% ...existing code...

\section*{Absolute Value Problems}

Find the absolute value of each number.

\begin{problem}
$|-8|$\\
$\answer{8}$
\begin{feedback}
The absolute value of a number is its distance from zero on the number line, regardless of direction. Therefore, $|-8| = 8$.
\end{feedback}
\end{problem}

\begin{problem}
$|5|$\\
$\answer{5}$
\begin{feedback}
The absolute value of a number is its distance from zero on the number line, regardless of direction. Therefore, $|5| = 5$.
\end{feedback}
\end{problem}

\begin{problem}
$|0|$\\
$\answer{0}$
\begin{feedback}
The absolute value of a number is its distance from zero on the number line, regardless of direction. Therefore, $|0| = 0$.
\end{feedback}
\end{problem}

\begin{problem}
$-|-13|$\\
$\answer{-13}$
\begin{feedback}
The absolute value of a number is its distance from zero on the number line, regardless of direction. Therefore, $|-13| = 13$. Since there is a negative sign in front of the absolute value, we have $-|-13| = -13$.
\end{feedback}
\end{problem}

% ...existing code...

\section*{Properties of Integers}

Properties of Integers (ERAU | 8:23)

\youtube{-a1kYA5q1Zs}

% filepath: /code/integers.tex
% ...existing code...

\section*{Adding and Subtracting Integers}

\begin{problem}
What is $6 + (-4)$?\\
$\answer{2}$
\begin{feedback}
When adding a negative number, think of it as subtracting the absolute value of that number. So, $6 + (-4)$ is the same as $6 - 4$, which equals $2$.
\end{feedback}
\end{problem}

\begin{problem}
What is $-5 + 7$?\\
$\answer{2}$
\begin{feedback}
When adding a positive number to a negative number, think of it as subtracting the absolute value of the negative number from the positive number. So, $-5 + 7$ is the same as $7 - 5$, which equals $2$.
\end{feedback}
\end{problem}

\begin{problem}
What is $-8 + (-3)$?\\
$\answer{-11}$
\begin{feedback}
When adding two negative numbers, add their absolute values and then put a negative sign in front of the result. So, $-8 + (-3)$ is the same as $-(8 + 3)$, which equals $-11$.
\end{feedback}
\end{problem}

\begin{problem}
Subtract: $9 - 12$\\
$\answer{-3}$
\begin{feedback}
When subtracting a larger number from a smaller number, the result is negative. So, $9 - 12 = -3$.
\end{feedback}
\end{problem}

\begin{problem}
Subtract: $-10 - (-6)$\\
$\answer{-4}$
\begin{feedback}
When subtracting a negative number, it is the same as adding its absolute value. So, $-10 - (-6)$ is the same as $-10 + 6$, which equals $-4$.
\end{feedback}
\end{problem}

\begin{problem}
Subtract: $3 - (-5)$\\
$\answer{8}$
\begin{feedback}
When subtracting a negative number, it is the same as adding its absolute value. So, $3 - (-5)$ is the same as $3 + 5$, which equals $8$.
\end{feedback}
\end{problem}

% ...existing code...

% filepath: /code/integers.tex
% ...existing code...

\section*{Multiplying and Dividing Integers}

\begin{problem}
Multiply: $5 \times (-4)$\\
$\answer{-20}$
\begin{feedback}
When multiplying a positive number by a negative number, the result is negative. So, $5 \times (-4) = -20$.
\end{feedback}
\end{problem}

\begin{problem}
Multiply: $-7 \times (-3)$\\
$\answer{21}$
\begin{feedback}
When multiplying two negative numbers, the result is positive. So, $-7 \times (-3) = 21$.
\end{feedback}
\end{problem}

\begin{problem}
Multiply: $0 \times 8$\\
$\answer{0}$
\begin{feedback}
Any number multiplied by zero is zero. So, $0 \times 8 = 0$.
\end{feedback}
\end{problem}

\begin{problem}
Divide: $-24 \div 6$\\
$\answer{-4}$
\begin{feedback}
When dividing a negative number by a positive number, the result is negative. So, $-24 \div 6 = -4$.
\end{feedback}
\end{problem}

\begin{problem}
Divide: $18 \div (-3)$\\
$\answer{-6}$
\begin{feedback}
When dividing a positive number by a negative number, the result is negative. So, $18 \div (-3) = -6$.
\end{feedback}
\end{problem}

\begin{problem}
Divide: $-30 \div (-5)$\\
$\answer{6}$
\begin{feedback}
When dividing two negative numbers, the result is positive. So, $-30 \div (-5) = 6$.
\end{feedback}
\end{problem}

% ...existing code...

% filepath: /code/integers.tex
% ...existing code...

\section*{Exponents with Positive and Negative Integers Practice}

\begin{problem}
Evaluate: $(-3)^2$\\
$\answer{9}$
\begin{feedback}
When raising a negative number to an even exponent, the result is positive. So, $(-3)^2 = 9$.
\end{feedback}
\end{problem}

\begin{problem}
Evaluate: $4^3$\\
$\answer{64}$
\begin{feedback}
When raising a positive number to any exponent, the result is positive. So, $4^3 = 64$.
\end{feedback}
\end{problem}

\begin{problem}
Evaluate: $-2^4$\\
$\answer{-16}$
\begin{feedback}
When raising a negative number to an odd exponent, the result is negative. So, $-2^4 = -16$.
\end{feedback}
\end{problem}

\begin{problem}
Evaluate: $(-5)^3$\\
$\answer{-125}$
\begin{feedback}
When raising a negative number to an odd exponent, the result is negative. So, $(-5)^3 = -125$.
\end{feedback}
\end{problem}

% ...existing code...

% filepath: /code/integers.tex
% ...existing code...

\section*{Orders of Operations Integer Problems}

\begin{problem}
Evaluate: $-2 \times [3 + (-5)]^2$\\
$\answer{-8}$
\begin{feedback}
First, solve inside the brackets: $3 + (-5) = -2$. Then, square the result: $(-2)^2 = 4$. Finally, multiply by $-2$: $-2 \times 4 = -8$.
\end{feedback}
\end{problem}

\begin{problem}
Simplify: $| -7 + 4 | - | 2 - 9 |$\\
$\answer{0}$
\begin{feedback}
First, solve inside the absolute value bars: $-7 + 4 = -3$ and $2 - 9 = -7$. Then, find the absolute values: $|-3| = 3$ and $|-7| = 7$. Finally, subtract the results: $3 - 7 = -4$.
\end{feedback}
\end{problem}

\begin{problem}
Evaluate: $(-2)^3 + 4 \times (-3)$\\
$\answer{-20}$
\begin{feedback}
First, evaluate the exponent: $(-2)^3 = -8$. Then, multiply: $4 \times (-3) = -12$. Finally, add the results: $-8 + (-12) = -20$.
\end{feedback}
\end{problem}

\begin{problem}
Simplify: $[(-6) \div 2]^2 - 5 \times (-2)$\\
$\answer{34}$
\begin{feedback}
First, divide inside the brackets: $(-6) \div 2 = -3$. Then, square the result: $(-3)^2 = 9$. Finally, subtract and multiply: $9 - 5 \times (-2) = 9 + 10 = 19$.
\end{feedback}
\end{problem}

\begin{problem}
Evaluate: $| -4 \times 3 | + (-2)^4 - 7$\\
$\answer{23}$
\begin{feedback}
First, solve inside the absolute value bars: $-4 \times 3 = -12$. Then, find the absolute value: $|-12| = 12$. Next, evaluate the exponent: $(-2)^4 = 16$. Finally, add and subtract: $12 + 16 - 7 = 23$.
\end{feedback}
\end{problem}

\begin{problem}
Simplify: $-5 + [2 \times (-3)]^2 \div (-6)$\\
$\answer{-7}$
\begin{feedback}
First, solve inside the brackets: $2 \times (-3) = -6$. Then, square the result: $(-6)^2 = 36$. Next, divide by $-6$: $36 \div (-6) = -6$. Finally, add and subtract: $-5 + (-6) = -11$.
\end{feedback}
\end{problem}

% ...existing code...

\end{document}