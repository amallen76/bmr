% filepath: /code/newActivityMod 1.tex
\documentclass{ximera}

\title{Decimals}
\author{Amy Riordan}

\begin{document}
\begin{abstract}
Decimal video resources from Module 1.
\end{abstract}
\maketitle

\section*{Decimals}

Below each video, you’ll find a set of practice problems. Try those problems first. If you run into any difficulties, go back and rewatch the video, then attempt the problems again. This way, you’ll reinforce what you’ve learned and build a stronger understanding.

\section*{Decimals}

Decimals (ERAU | 6:02)

\youtube{Ucq2060ILFI}

% filepath: /code/decimals.tex
% ...existing code...

\section*{Finding the Place Value}
\begin{problem}
What is the place value of the digit $5$ in $3.452$?\\
$\answer{hundredths}$
\begin{feedback}
The digit $5$ is in the second position to the right of the decimal point, which represents the hundredths place.
\end{feedback}
\end{problem}

\begin{problem}
What is the place value of the digit $7$ in $0.0781$?\\
$\answer{thousandths}$
\begin{feedback}
The digit $7$ is in the third position to the right of the decimal point, which represents the thousandths place.
\end{feedback}
\end{problem}

\section*{Writing decimals in word form}
\begin{problem}
Write $0.56$ in word form.\\
$\answer{fifty-six hundredths}$
\begin{feedback}
The number $0.56$ is read as "fifty-six hundredths."
\end{feedback}
\end{problem}

\begin{problem}
Write $3.204$ in word form.\\
$\answer{three and two hundred four thousandths}$
\begin{feedback}
The number $3.204$ is read as "three and two hundred four thousandths."
\end{feedback}
\end{problem}

\section*{Writing Decimals in Standard Form}

\begin{problem}
Write "nine and thirty-two hundredths" in standard form.\\
$\answer{9.32}$
\begin{feedback}
The phrase "nine and thirty-two hundredths" translates to the number $9.32$ in standard form.
\end{feedback}
\end{problem}

\begin{problem}
Write "one hundred five thousandths" in standard form.\\
$\answer{0.105}$
\begin{feedback}
The phrase "one hundred five thousandths" translates to the number $0.105$ in standard form.
\end{feedback}
\end{problem}

\section*{Writing Decimals as Simplified Fractions}

\begin{problem}
Write $0.8$ as a simplified fraction.\\
$\answer{\frac{4}{5}}$
\begin{feedback}
To convert $0.8$ to a fraction, write it as $\frac{8}{10}$ and then simplify by dividing both the numerator and denominator by their greatest common divisor, which is $2$. This gives $\frac{4}{5}$.
\end{feedback}
\end{problem}

\begin{problem}
Write $0.125$ as a simplified fraction.\\
$\answer{\frac{1}{8}}$
\begin{feedback}
To convert $0.125$ to a fraction, write it as $\frac{125}{1000}$ and then simplify by dividing both the numerator and denominator by their greatest common divisor, which is $125$. This gives $\frac{1}{8}$.
\end{feedback}
\end{problem}

% ...existing code...

\section*{Rounding Decimals}

Rounding Decimals (ERAU | 3:51)

\youtube{3YjnFSi_Vxo}

\section*{Rounding Decimals}

\begin{problem}
Round $4.678$ to the nearest tenth.\\
$\answer{4.7}$
\begin{feedback}
To round $4.678$ to the nearest tenth, look at the digit in the hundredths place, which is $7$. Since $7$ is greater than or equal to $5$, round up the tenths place from $6$ to $7$. The rounded number is $4.7$.
\end{feedback}
\end{problem}

\begin{problem}
Round $0.9342$ to the nearest hundredth.\\
$\answer{0.93}$
\begin{feedback}
To round $0.9342$ to the nearest hundredth, look at the digit in the thousandths place, which is $4$. Since $4$ is less than $5$, keep the hundredths place as it is. The rounded number is $0.93$.
\end{feedback}
\end{problem}

\section*{Identifying Place Value}

\begin{problem}
What is the place value of the digit $6$ in $45.628$?\\
$\answer{hundredths}$
\begin{feedback}
The digit $6$ is in the second position to the right of the decimal point, which represents the hundredths place.
\end{feedback}
\end{problem}

\begin{problem}
What is the place value of the digit $2$ in $0.2047$?\\
$\answer{tenths}$
\begin{feedback}
The digit $2$ is in the first position to the right of the decimal point, which represents the tenths place.
\end{feedback}
\end{problem}

\begin{problem}
What is the place value of the digit $9$ in $13.095$?\\
$\answer{hundredths}$
\begin{feedback}
The digit $9$ is in the second position to the right of the decimal point, which represents the hundredths place.
\end{feedback}
\end{problem}

\section*{Adding Decimals}

Adding Decimals (ERAU | 3:56)

\youtube{qgWbs-97brU}

% filepath: /code/decimals.tex
% ...existing code...

\section*{Adding Decimals}

\begin{problem}
Add: $2.35 + 4.17 = \answer{6.52}$
\begin{feedback}
To add $2.35$ and $4.17$, align the decimal points and add the numbers column by column:
\begin{align*}
  2.35 \\
+ 4.17 \\
\hline
  6.52
\end{align*}
The sum is $6.52$.
\end{feedback}
\end{problem}

\begin{problem}
Add: $0.75 + 1.28 = \answer{2.03}$
\begin{feedback}
To add $0.75$ and $1.28$, align the decimal points and add the numbers column by column:
\begin{align*}
  0.75 \\
+ 1.28 \\
\hline
  2.03
\end{align*}
The sum is $2.03$.
\end{feedback}
\end{problem}

\begin{problem}
Add: $5.6 + 3.45 = \answer{9.05}$
\begin{feedback}
To add $5.6$ and $3.45$, align the decimal points and add the numbers column by column:
\begin{align*}
  5.60 \\
+ 3.45 \\
\hline
  9.05
\end{align*}
The sum is $9.05$.
\end{feedback}
\end{problem}

\begin{problem}
Add: $12.09 + 0.91 = \answer{13.00}$
\begin{feedback}
To add $12.09$ and $0.91$, align the decimal points and add the numbers column by column:
\begin{align*}
  12.09 \\
+  0.91 \\
\hline
  13.00
\end{align*}
The sum is $13.00$.
\end{feedback}
\end{problem}

\begin{problem}
Add: $0.007 + 0.083 = \answer{0.09}$
\begin{feedback}
To add $0.007$ and $0.083$, align the decimal points and add the numbers column by column:
\begin{align*}
  0.007 \\
+ 0.083 \\
\hline
  0.090
\end{align*}
The sum is $0.09$.
\end{feedback}
\end{problem}

% ...existing code...

\section*{Subtracting Decimals}

Subtracting Decimals (ERAU | 6:43)

\youtube{rKddVA1esAU}

% filepath: /code/decimals.tex
% ...existing code...

\section*{Subtracting Decimals} 

\begin{problem}
Subtract: $7.85 - 2.43 = \answer{5.42}$
\begin{feedback}
To subtract $2.43$ from $7.85$, align the decimal points and subtract the numbers column by column:
\begin{align*}
  7.85 \\
- 2.43 \\
\hline
  5.42
\end{align*}
The difference is $5.42$.
\end{feedback}
\end{problem}

\begin{problem}
Subtract: $10.5 - 4.76 = \answer{5.74}$
\begin{feedback}
To subtract $4.76$ from $10.5$, align the decimal points and subtract the numbers column by column:
\begin{align*}
 10.50 \\
- 4.76 \\
\hline
  5.74
\end{align*}
The difference is $5.74$.
\end{feedback}
\end{problem}

\begin{problem}
Subtract: $3.21 - 1.09 = \answer{2.12}$
\begin{feedback}
To subtract $1.09$ from $3.21$, align the decimal points and subtract the numbers column by column:
\begin{align*}
  3.21 \\
- 1.09 \\
\hline
  2.12
\end{align*}
The difference is $2.12$.
\end{feedback}
\end{problem}

\begin{problem}
Subtract: $6.03 - 0.58 = \answer{5.45}$
\begin{feedback}
To subtract $0.58$ from $6.03$, align the decimal points and subtract the numbers column by column:
\begin{align*}
  6.03 \\
- 0.58 \\
\hline
  5.45
\end{align*}
The difference is $5.45$.
\end{feedback}
\end{problem}

\begin{problem}
Subtract: $0.9 - 0.27 = \answer{0.63}$
\begin{feedback}
To subtract $0.27$ from $0.9$, align the decimal points and subtract the numbers column by column:
\begin{align*}
  0.90 \\
- 0.27 \\
\hline
  0.63
\end{align*}
The difference is $0.63$.
\end{feedback}
\end{problem}

% ...existing code...

\section*{Multiplying Decimals}

Multiplying Decimals (ERAU | 7:59)

\youtube{7dVmEuWB1A8}

% filepath: /code/decimals.tex
% ...existing code...

\section*{Multiplying Decimals}

\begin{problem}
Multiply: $2.5 \times 0.4 = \answer{1.0}$
\begin{feedback}
To multiply $2.5$ by $0.4$, first ignore the decimal points and multiply the numbers as if they were whole numbers:
\begin{align*}
  25 \\
\times 4 \\
\hline
  100
\end{align*}
Next, count the total number of decimal places in the original numbers (2 + 1 = 3) and place the decimal point in the product so that there are three decimal places:
\begin{align*}
  1.00
\end{align*}
The product is $1.0$.
\end{feedback}
\end{problem}

\begin{problem}
Multiply: $3.6 \times 1.2 = \answer{4.32}$
\begin{feedback}
To multiply $3.6$ by $1.2$, first ignore the decimal points and multiply the numbers as if they were whole numbers:
\begin{align*}
   36 \\
\times   12 \\
\hline
   72 \\
+360 \\
\hline
+432
\end{align*}
Next, count the total number of decimal places in the original numbers (1 + 1 = 2) and place the decimal point in the product so that there are two decimal places:
\begin{align*}
   4.32
\end{align*}
The product is $4.32$.
\end{feedback}
\end{problem}

\begin{problem}
Multiply: $0.75 \times 0.8 = \answer{0.6}$
\begin{feedback}
To multiply $0.75$ by $0.8$, first ignore the decimal points and multiply the numbers as if they were whole numbers:
\begin{align*}
   75 \\
\times   8 \\
\hline
   600
\end{align*}
Next, count the total number of decimal places in the original numbers (2 + 1 = 3) and place the decimal point in the product so that there are three decimal places:
\begin{align*}
   0.600
\end{align*}
The product is $0.6$.
\end{feedback}
\end{problem}

\begin{problem}
Multiply: $5.25 \times 0.2 = \answer{1.05}$
\begin{feedback}
To multiply $5.25$ by $0.2$, first ignore the decimal points and multiply the numbers as if they were whole numbers:
\begin{align*}
   525 \\
\times   2 \\
\hline
   1050
\end{align*}
Next, count the total number of decimal places in the original numbers (2 + 1 = 3) and place the decimal point in the product so that there are three decimal places:
\begin{align*}
   1.050
\end{align*}
The product is $1.05$.
\end{feedback}
\end{problem}

\begin{problem}
Multiply: $7.1 \times 0.3 = \answer{2.13}$
\begin{feedback}
To multiply $7.1$ by $0.3$, first ignore the decimal points and multiply the numbers as if they were whole numbers:
\begin{align*}
   71 \\
\times   3 \\
\hline
   213
\end{align*}
Next, count the total number of decimal places in the original numbers (1 + 1 = 2) and place the decimal point in the product so that there are two decimal places:
\begin{align*}
   2.13
\end{align*}
The product is $2.13$.
\end{feedback}
\end{problem}

% ...existing code...

\section*{Dividing Decimals}

Dividing Decimals (Part 1) (ERAU | 8:28)

\youtube{3WpVWne4R0Y}

Dividing Decimals (Part 2) (ERAU | 3:39)

\youtube{YNedH0TpAOg}

% filepath: /code/decimals.tex
% ...existing code...

\section*{Dividing Decimals}

\begin{problem}
Divide: $6.4 \div 0.8 = \answer{8}$
\begin{feedback}
To divide $6.4$ by $0.8$, move the decimal point in the divisor to the right until it becomes a whole number, and move the decimal point in the dividend the same number of places to the right:
\begin{align*}
  64 \\
\div 8 \\
\hline
  8
\end{align*}
The quotient is $8$.
\end{feedback}
\end{problem}

\begin{problem}
Divide: $3.75 \div 1.5 = \answer{2.5}$
\begin{feedback}
To divide $3.75$ by $1.5$, move the decimal point in the divisor to the right until it becomes a whole number, and move the decimal point in the dividend the same number of places to the right:
\begin{align*}
  37.5 \\
\div 15 \\
\hline
  2.5
\end{align*}
The quotient is $2.5$.
\end{feedback}
\end{problem}

\begin{problem}
Divide: $12.6 \div 0.3 = \answer{42}$
\begin{feedback}
To divide $12.6$ by $0.3$, move the decimal point in the divisor to the right until it becomes a whole number, and move the decimal point in the dividend the same number of places to the right:
\begin{align*}
  126 \\
\div 3 \\
\hline
  42
\end{align*}
The quotient is $42$.
\end{feedback}
\end{problem}

\begin{problem}
Divide: $0.56 \div 0.07 = \answer{8}$
\begin{feedback}
To divide $0.56$ by $0.07$, move the decimal point in the divisor to the right until it becomes a whole number, and move the decimal point in the dividend the same number of places to the right:
\begin{align*}
  56 \\
\div 7 \\
\hline
  8
\end{align*}
The quotient is $8$.
\end{feedback}
\end{problem}

\begin{problem}
Divide: $9.9 \div 3 = \answer{3.3}$
\begin{feedback}
To divide $9.9$ by $3$, move the decimal point in the divisor to the right until it becomes a whole number, and move the decimal point in the dividend the same number of places to the right:
\begin{align*}
  99 \\
\div 30 \\
\hline
  3.3
\end{align*}
The quotient is $3.3$.
\end{feedback}
\end{problem}

% ...existing code...

\end{document}