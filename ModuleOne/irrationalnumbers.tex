\documentclass{ximera}

\title{Irrational Numbers}
\author{Amy Riordan}

\begin{document}
\begin{abstract}
Irrational Numbers video resources from Module 1.
\end{abstract}
\maketitle

\section*{Irrational Numbers}

Below each video, you’ll find a set of practice problems. Try those problems first. If you run into any difficulties, go back and rewatch the video, then attempt the problems again. This way, you’ll reinforce what you’ve learned and build a stronger understanding.

\section*{Multiplying Irrational Numbers}

Multiplying Irrational Numbers (ERAU|4:28)

\youtube{placeholder}

% filepath: /code/irrationalnumbers.tex

% filepath: /code/irrationalnumbers.tex
\section*{Simplifying Square Roots}

\begin{problem}
$\sqrt{18} = \answer{3\sqrt{2}}$

\begin{feedback}
Factor $18$ as $9 \times 2$. $\sqrt{18} = \sqrt{9 \times 2} = \sqrt{9} \times \sqrt{2} = 3\sqrt{2}$.
\end{feedback}
\end{problem}

\begin{problem}
$\sqrt{50} = \answer{5\sqrt{2}}$

\begin{feedback}
Factor $50$ as $25 \times 2$. $\sqrt{50} = \sqrt{25 \times 2} = \sqrt{25} \times \sqrt{2} = 5\sqrt{2}$.
\end{feedback}
\end{problem}

\begin{problem}
$\sqrt{32} = \answer{4\sqrt{2}}$

\begin{feedback}
Factor $32$ as $16 \times 2$. $\sqrt{32} = \sqrt{16 \times 2} = \sqrt{16} \times \sqrt{2} = 4\sqrt{2}$.
\end{feedback}
\end{problem}

\begin{problem}
$\sqrt{72} = \answer{6\sqrt{2}}$

\begin{feedback}
Factor $72$ as $36 \times 2$. $\sqrt{72} = \sqrt{36 \times 2} = \sqrt{36} \times \sqrt{2} = 6\sqrt{2}$.
\end{feedback}
\end{problem}

\begin{problem}
$\sqrt{12} = \answer{2\sqrt{3}}$

\begin{feedback}
Factor $12$ as $4 \times 3$. $\sqrt{12} = \sqrt{4 \times 3} = \sqrt{4} \times \sqrt{3} = 2\sqrt{3}$.
\end{feedback}
\end{problem}

\section*{Multiplying Irrational Numbers}

\begin{problem}
$\sqrt{2} \times \sqrt{3} = \answer{\sqrt{6}}$

\begin{feedback}
When multiplying square roots, multiply the numbers inside: $\sqrt{2} \times \sqrt{3} = \sqrt{2 \times 3} = \sqrt{6}$.
\end{feedback}
\end{problem}

\begin{problem}
$\sqrt{5} \times 2 = \answer{2\sqrt{5}}$

\begin{feedback}
Multiply the integer by the irrational number: $2 \times \sqrt{5} = 2\sqrt{5}$.
\end{feedback}
\end{problem}

\begin{problem}
$3\sqrt{7} \times 4 = \answer{12\sqrt{7}}$

\begin{feedback}
Multiply the coefficients: $3 \times 4 = 12$, so $3\sqrt{7} \times 4 = 12\sqrt{7}$.
\end{feedback}
\end{problem}

\begin{problem}
$\sqrt{2} \times \sqrt{8} = \answer{4}$

\begin{feedback}
$\sqrt{2} \times \sqrt{8} = \sqrt{2 \times 8} = \sqrt{16} = 4$.
\end{feedback}
\end{problem}

\begin{problem}
$2\sqrt{3} \times 5\sqrt{3} = \answer{30}$

\begin{feedback}
Multiply the coefficients: $2 \times 5 = 10$. Multiply the square roots: $\sqrt{3} \times \sqrt{3} = \sqrt{9} = 3$. So, $10 \times 3 = 30$.
\end{feedback}
\end{problem}

\section*{Dividing Irrational Numbers}

Dividing Irrational Numbers (ERAU|1:16)

\youtube{placeholder}

% filepath: /code/irrationalnumbers.tex
\section*{Dividing Irrational Numbers}

\begin{problem}
$\frac{\sqrt{12}}{\sqrt{3}} = \answer{2}$

\begin{feedback}
Divide the numbers inside the square roots: $\frac{\sqrt{12}}{\sqrt{3}} = \sqrt{\frac{12}{3}} = \sqrt{4} = 2$.
\end{feedback}
\end{problem}

\begin{problem}
$\frac{6\sqrt{5}}{2} = \answer{3\sqrt{5}}$

\begin{feedback}
Divide the coefficients: $6 \div 2 = 3$. The irrational part stays the same, so $6\sqrt{5} \div 2 = 3\sqrt{5}$.
\end{feedback}
\end{problem}

\begin{problem}
$\frac{\sqrt{18}}{\sqrt{2}} = \answer{3}$

\begin{feedback}
$\frac{\sqrt{18}}{\sqrt{2}} = \sqrt{\frac{18}{2}} = \sqrt{9} = 3$.
\end{feedback}
\end{problem}

\begin{problem}
$\frac{8\sqrt{3}}{4\sqrt{3}} = \answer{2}$

\begin{feedback}
Divide the coefficients: $8 \div 4 = 2$. Divide the square roots: $\sqrt{3} \div \sqrt{3} = 1$. So, $2 \times 1 = 2$.
\end{feedback}
\end{problem}

\begin{problem}
$\frac{\sqrt{50}}{5} = \answer{2}$

\begin{feedback}
$\sqrt{50} = \sqrt{25 \times 2} = 5\sqrt{2}$. So, $\frac{5\sqrt{2}}{5} = \sqrt{2}$. But $\frac{\sqrt{50}}{5} = \frac{5\sqrt{2}}{5} = \sqrt{2}$.
\end{feedback}
\end{problem}

\section*{Adding and Subtracting Irrational Numbers}

Adding and Subtracting Irrational Numbers (ERAU|3:02)

\youtube{-xM3QUOTSVE}

% filepath: /code/irrationalnumbers.tex
\section*{Adding and Subtracting Irrational Numbers}

\begin{problem}
$\sqrt{5} + \sqrt{5} = \answer{2\sqrt{5}}$

\begin{feedback}
Add the coefficients: $1 + 1 = 2$. The irrational part stays the same, so $\sqrt{5} + \sqrt{5} = 2\sqrt{5}$.
\end{feedback}
\end{problem}

\begin{problem}
$3\sqrt{2} - \sqrt{2} = \answer{2\sqrt{2}}$

\begin{feedback}
Subtract the coefficients: $3 - 1 = 2$. The irrational part stays the same, so $3\sqrt{2} - \sqrt{2} = 2\sqrt{2}$.
\end{feedback}
\end{problem}

\begin{problem}
$\sqrt{7} + 2\sqrt{7} = \answer{3\sqrt{7}}$

\begin{feedback}
Add the coefficients: $1 + 2 = 3$. The irrational part stays the same, so $\sqrt{7} + 2\sqrt{7} = 3\sqrt{7}$.
\end{feedback}
\end{problem}

\begin{problem}
$5\sqrt{3} - 2\sqrt{3} = \answer{3\sqrt{3}}$

\begin{feedback}
Subtract the coefficients: $5 - 2 = 3$. The irrational part stays the same, so $5\sqrt{3} - 2\sqrt{3} = 3\sqrt{3}$.
\end{feedback}
\end{problem}

\begin{problem}
$\sqrt{2} + \sqrt{3}$ = $\answer{\sqrt{2} + \sqrt{3}}$

\begin{feedback}
You cannot combine $\sqrt{2}$ and $\sqrt{3}$ because they are not like terms. The answer is $\sqrt{2} + \sqrt{3}$.
\end{feedback}
\end{problem}

\section*{Rationalizing Denominators}

Rationalizing Denominators (ERAU|4:04)

\youtube{placeholder}

\section*{Rationalizing Denominators}

% filepath: /code/irrationalnumbers.tex
\section*{Practice: Rationalizing the Denominator}

\begin{problem}
$\frac{1}{\sqrt{2}} = \answer{\frac{\sqrt{2}}{2}}$

\begin{feedback}
Multiply numerator and denominator by $\sqrt{2}$: $\frac{1}{\sqrt{2}} \times \frac{\sqrt{2}}{\sqrt{2}} = \frac{\sqrt{2}}{2}$.
\end{feedback}
\end{problem}

\begin{problem}
$\frac{3}{\sqrt{5}} = \answer{\frac{3\sqrt{5}}{5}}$

\begin{feedback}
Multiply numerator and denominator by $\sqrt{5}$: $\frac{3}{\sqrt{5}} \times \frac{\sqrt{5}}{\sqrt{5}} = \frac{3\sqrt{5}}{5}$.
\end{feedback}
\end{problem}

\begin{problem}
${\sqrt{\frac{2}{3}}} = \answer{\frac{\sqrt{6}}{3}}$

\begin{feedback}
Multiply numerator and denominator by $\sqrt{3}$: $\sqrt{\frac{2}{3}} = \frac{\sqrt{2} \times \sqrt{3}}{\sqrt{3} \times \sqrt{3}} = \frac{\sqrt{6}}{3}$.
\end{feedback}
\end{problem}

\begin{problem}
$\frac{5}{\sqrt{7}} = \answer{\frac{5\sqrt{7}}{7}}$

\begin{feedback}
Multiply numerator and denominator by $\sqrt{7}$: $\frac{5}{\sqrt{7}} \times \frac{\sqrt{7}}{\sqrt{7}} = \frac{5\sqrt{7}}{7}$.
\end{feedback}
\end{problem}

\begin{problem}
$\frac{4}{\sqrt{8}} = \answer{\frac{2\sqrt{2}}{4}}$

\begin{feedback}
First, multiply numerator and denominator by $\sqrt{8}$: $\frac{4}{\sqrt{8}} \times \frac{\sqrt{8}}{\sqrt{8}} = \frac{4\sqrt{8}}{8}$. Simplify $\sqrt{8} = 2\sqrt{2}$, so $\frac{4 \times 2\sqrt{2}}{8} = \frac{8\sqrt{2}}{8} = \sqrt{2}$. Alternatively, $\frac{4}{\sqrt{8}} = \frac{2\sqrt{2}}{4}$.
\end{feedback}
\end{problem}

\end{document}
